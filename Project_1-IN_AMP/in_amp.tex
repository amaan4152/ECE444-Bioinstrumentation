\documentclass{article}
\usepackage[utf8]{inputenc}
\usepackage{amsmath}
\usepackage{amsfonts}
\usepackage{subcaption}
\usepackage{multirow}
\usepackage{siunitx}
\usepackage{parskip}
\usepackage{marvosym}
\usepackage{mathtools}
\usepackage{graphicx} % Include figure files
\usepackage{bm}
\usepackage{booktabs}
\usepackage{hyperref}
\usepackage{pdfpages}
\usepackage[T1]{fontenc}
\usepackage{baskervillef}
\usepackage[varqu,varl,var0]{inconsolata}
\usepackage[scale=.95,type1]{cabin}
\usepackage[baskerville,vvarbb]{newtxmath}
\usepackage[cal=boondoxo]{mathalfa}
\usepackage{natbib}
\usepackage{graphicx}
\usepackage{librebaskerville}
\usepackage[T1]{fontenc}
\graphicspath{./images/}
\title{Instrumentation Amplifier}
\author{Amaan Rahman \\
The Cooper Union \\
B.E. of Electrical Engineering \\
ECE-444: Bio-instrumentation}

\begin{document}
\maketitle
\section{Abstract}
The instrumentation amplifier is 2-stage amplifier; the former stage consists of 2 differential operation amplifiers configured as buffer amplifiers, and the latter stage consists of a unity gain amplifier. The advantages of this amplifier topology are high open loop gain and high Common-Mode-Rejection-Ratio (CMMR). Thus, the instrumentation amplifier is practical in scenarios where a differential signal is small and further amplification is required. 

\section{Introduction}
An LTSpice simulation schematic has been setup, as can be seen in (FIGURE). The physical circuit has been built accordingly to the topology structure in (FIGURE). \\

The gain of the circuit can be derived as follows:
\begin{equation*}
    \begin{aligned}
        \Delta v_{in} &= v_{+in} - v_{-in} \\
    \end{aligned}
\end{equation*}
\[
    \begin{rcases}
        \frac{v_{out} - v_y}{R} &= \frac{v_y - v_{o_1}}{R}
        
    \end{rcases}
\]
\end{document}

